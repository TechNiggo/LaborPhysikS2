\documentclass{./setup/class} %Dokumentenart


%Pakete und Erweiterungen
\usepackage[german]{babelbib}
%Diagramme
\usepackage{pgfplots}
	\usepgfplotslibrary{colorbrewer}
	\usepgfplotslibrary{units}
%Schaltungen
\usepackage{tikz}
	\usetikzlibrary{calc, arrows, intersections}
    \usetikzlibrary{angles,quotes,babel}
	\usetikzlibrary{shapes,snakes,positioning}
	\usetikzlibrary{external, backgrounds} 
	\usetikzlibrary{pgfplots.polar} 
	%\usetikzlibrary{decorations.markings}
	%\tikzexternalize
	\tikzset{
	maxwidth/.style=	{text width=0.25\linewidth},
	wenn/.style=		{maxwidth,	draw, fill=orange!50, diamond, aspect=2.5},
	start/.style=		{maxwidth,	draw, fill=blue!30, rectangle,rounded corners},
	ausgabe/.style=		{maxwidth, 	draw, fill=red!30, trapezium, trapezium left angle=75, trapezium right angle=-75},
	arrow/.style=		{->},
	arrowlab/.style=	{pos=0.1,sloped,above, rotate=0},
	arrowlabmid/.style=	{pos=0.1,sloped,right, rotate=90},
	unwichtig/.style=	{}
	
	}
	\tikzset{
    	partial ellipse/.style args={#1:#2:#3}{
       		insert path={+ (#1:#3) arc (#1:#2:#3)}
    	}
	}
%Schaltungen
\usepackage[european, straightvoltages, americaninductors, siunitx,smartlabels]{circuitikz} 
%Farben (rowcolor)
\usepackage{colortbl}
%Farben
\usepackage{color}
%Tabellen über mehrere Seiten
\usepackage{longtable}
%Csv Dateien in Tabelle
\usepackage{csvsimple}
%Zellen in Tabelle vertical verbinden
\usepackage{multirow}
\usepackage{transparent}
%Tables und Figures an seinen Platz erzwungen
\usepackage{float}
%Vektorgraiken
\graphicspath{{}} %Pfad für Vektorgrafiken
%Kapitelweise durchnummerieren
\usepackage{chngcntr}
	\counterwithin{figure}{section}
	\counterwithin{table}{section}
	\counterwithin{equation}{section}
%Abkürzungen
\usepackage[printonlyused]{acronym}
%leerzeichen zwischen einheit und zahl, automatisch runden
\usepackage{siunitx}
	\sisetup{locale = DE, exponent-product=\cdot, round-mode=places, round-precision=2, output-product=\cdot, input-product=*, round-integer-to-decimal, round-minimum = 0.01}
%Inhaltsverzeichnis verlinkt
\usepackage{hyperref}
	\hypersetup{
		colorlinks,
    	citecolor=black,
    	filecolor=black,
   	 	linkcolor=black,
    	urlcolor=black,
    	linktoc=all,     %set to all if you want both sections and subsections linked
	}
%Eigene Bibliothek
%==============================================%
%==============Diagrammdefinition==============%
%Autor: 		Nico Herkner
%Erstelldatum:	
%Version:
%		1.2		2017-11-01	NH
%			NH	added:	\leereZellen[2]
%			NH	added:	\kariert[2]
%			NH	added:	\linie[1]
%			NH	changed:\linie[2] 
%			NH	changed:\leereZellen
%		1.3		2017-11-04	NH
%			NH	changed:\data
%			NH	changed:\data
%			NH	added:	\pgfplotsset
%			NH	added:	\MyAxis style
%			NH	added:	\MyPlotCsv style
%		1.4		2017-11-15	NH
%			NH	added:	\ome
%			NH	added:	\grad
%			NH	added:	Längen


%Settings for Tikz and pgfplots
\pgfplotsset{
	MyAxis/.style={
		/pgf/number format/.cd, use comma,
		1000 sep={.},
		anchor=origin,	% Align the origins
		%x=1cm, y=1cm,	% Set the same unit vectors
		grid=major, % erstellt ein Raster
		width=\linewidth,
		cycle list name=MyCyclelist,
	},
	MyPlotCsv/.style={
		/pgf/number format/read comma as period, %Komma statt Punkt als Dezimaltrennzeichen
		domain=0:360,
		samples=100
	},
}


%Counter
\newcounter{owncsvrow}
\newcounter{counterA}
\newcounter{counterB}
\newcounter{counterC}
\newcounter{counterD}
\newcounter{counterE}
\newcounter{counterF}

%neue Definitionen
\newcommand{\ome}{\omega}
\newcommand{\grad}{^\circ}
\newcommand{\leer}[0]{ } % Leertaste
\newcommand{\fett}[1]{\textbf{#1}}%Fett
\newcommand{\kursiv}[1]{\\textit{#1}}%Kursiv
\newcommand{\unterstrichen}[1]{\textbf{#1}}%Unterstruchen
%Diagramm
\newcommand{\data}[3][]{
	\addplot
		table[
			col sep =semicolon, %Trennzeichen zwischen Spalten
			/pgf/number format/read comma as period, %Komma statt Punkt als Dezimaltrennzeichen
			y=#3, %Spaltenname für y-Werte
			domain=0:15,
    		samples=100,
			#1
		]{#2};
}
\newcommand{\diagramm}[3]{
	\begin{tikzpicture}[show background rectangle]
		\begin{axis}[
			%smooth,
			%axis x line=bottom, axis y line=left, %Pfeile bei den Achsen
			width=\linewidth, %passt die breite des Diagramms auf die Seitenbreite an
			grid=major, % erstellt ein Raster
			grid style={dashed,gray!30}, %formatiert das Raster
			cycle list name=MyCyclelist,
			xlabel=$#1$\newline test, %Erstellt eine Achsenbeschriftung
			ylabel=$#2$,
			%legend style={at={(0.5,-0.2)},anchor=north}, %Positioniert die Legende unter der Tabelle
			]
		#3 %Daten
		\end{axis}
	\end{tikzpicture}
}
%Leere Zeilen in einer Tabelle \leereZeilen{ZeilenAnzahl}{Aufbau einer Zeile mit &}
\newcommand{\leereZellen}[2]{
	\hline
	\csvreader[separator = semicolon, late after line =\\\hline, filter test=\ifnumless{\csvcoli}{#1+1}]{./tables/hundert.csv}{}{#2}
}
%Karriert \kariert[steplength]{zeilen}
\newcommand\kariert[2][0.5cm]{% 
   \begin{tikzpicture}[gray,step=#1]
     \pgfmathtruncatemacro\anzahl{(\linewidth-\pgflinewidth)/#1} % maximale Anzahl Kästchen pro Zeile
     \draw (0,0) rectangle (\anzahl*#1,#2*#1) (0,0) grid (\anzahl*#1,#2*#1);
   \end{tikzpicture} 
}
\newcommand{\linie}[2][\textwidth]{
	\foreach \x in {1,...,#2} {\\[9pt] \noindent\rule{#1}{1pt}}
}

%Ergebnis doppelt unterstrichen
\newcommand{\result}[1]{
	\underline{\underline{#1}}
}
%Einheit
\newcommand{\unit}[1]{
	~\mathrm{#1}
}
%Unterstrichen
\renewcommand{\u}[1]{
	\underline{#1}
}
%Bruch
\renewcommand{\/}[2]{
	~\frac{#1}{#2}
}
%Bruch
\newcommand{\fraq}[2]{
	~\frac{#1}{#2}
}
%Gleichung
\newcommand{\eq}[1]{
	(\ref{#1})
}
%Varphi
\newcommand{\fii}[0]{
	\varphi \alpha
}
%Vektor
\newcommand{\vek}[1]{
	\overset{\rightharpoonup}{#1}
}
%Zahlenmengen
\newcommand{\R}{\mathbb{R}}
\newcommand{\C}{\mathbb{C}}
\newcommand{\N}{\mathbb{N}}
\newcommand{\G}{\mathbb{G}}
\newcommand{\Q}{\mathbb{Q}}
%Betragsstriche
\providecommand{\abs}[1]{
	\lvert #1 \rvert
}


%Neue Umgebungen
%Tabelle
\newenvironment{tabele}{\begin{longtable}}{\end{longtable}}
\newenvironment{tabelle}{\begin{table}[H] \centering}{\end{table}}
\newenvironment{abbildung}{\begin{figure}[H] \centering}{\end{figure}}


%Neue Farben
\definecolor{darkspringgreen}{rgb}{0.09, 0.45, 0.27}
\definecolor{darkpastelred}{rgb}{0.76, 0.23, 0.13}
\definecolor{dimgray}{rgb}{0.41, 0.41, 0.41}
\definecolor{firstcsvrow}{rgb}{0.68, 0.68, 1}
\definecolor{secondcsvrow}{rgb}{0.75, 0.75, 0.75}

%Neuer Spaltentyp 
\newcolumntype{K}[1]{>{\centering\arraybackslash}p{#1}} %Mittig mit bestimmter Breie

%Farbreihenfolge bei Plots
\pgfplotscreateplotcyclelist{MyCyclelist}{%
  {red, mark = *, thick},
  {blue, mark = square*, thick},
  {orange, mark = diamond*, thick},
  {green, mark = pentagon*, thick},
  {yellow, mark = *, thick},
  {black, mark = triangle*, thick},
}
%legend image post style={line width=1pt}

%Längen
\newlength{\tablewidth}
\setlength{\tablewidth}{1.5cm}
\setlength{\parindent}{0em} %einrücken verhindern
%Kopf- und Fusszeile
%Dokumentendaten

\renewcommand{\labor}{Physik} %Laborname
\renewcommand{\versuch}{S2} %Versuchnummer
\renewcommand{\versuchtitel}{Untersuchung von erzwungenen Schwingungen mit einem Linear-Pendel} %Versuchtitel
\renewcommand{\betreuer}{Frau Dr. Nicolaus} %Versuchtitel
\renewcommand{\datum}{17. November 2017} %Datum
\author{Marius Neumann \and \&  \and Nico Herkner}
%Längen
\newlength{\tablewidth}
\setlength{\tablewidth}{1.5cm}
\setlength{\parindent}{0em} %einrücken verhindern

%Literaturveichnis
\bibliographystyle{babplain-lf}
\begin{document}
\newgeometry{includehead, ignorefoot, bindingoffset=10mm, top=17mm, left=17mm, right=17mm, bottom=17mm}
\maketitle
\tableofcontents
\thispagestyle{titelseite}
\setcounter{page}{0}
\restoregeometry
%======================================%
%================Inhalt================%
\section{Theorie}
\label{sec:theorie}
Wir haben uns anhand \cite{Bleckwedel.Mathe2,Dorn.2008,Turtur.Sommersemester2017} mit den genannten Themen vertraut gemacht und uns die Versuchsanleitung sorgfältig durchgelesen. Uns sind keine weiteren Fragen offengeblieben.
Des Weiteren haben wir uns die Vorbereitungsstichpunkte genau angeschaut.
\section{Vorbereitung}
\label{sec:vorbereitung}
	\newpage
\section{Versuchsdurchführung}
\label{sec:durchfuehrung}
Wir haben die Messung, wie im Laborumdruck beschrieben, für vier verschiedene Messpunkte durchgeführt und die Messwerte im Messprotokoll eingetragen. Siehe Tabelle \ref{tab:messprotokoll->temperatur} bis \ref{tab:messprotokoll->messung_4}.
	\newpage
\section{Auswertung}
\label{sec:auswertung}

%==========Nico=================
\subsection{Berechnung von $\overline{\Delta l_n}$}
\label{sub:auswertung->MittelwertResonanzabstand}
Die Tabellen \ref{tab:auswertung->MittelwertResonanzabstand->Messung1} bis \ref{tab:auswertung->MittelwertResonanzabstand->Messung4} zeigen unsere Messergebnisse mit den dazu gehörigen Differenzen. $\Delta l_n$ wurde wie folgt berechnet:
\begin{align}
	\Delta l_n &= l_{max,n} - l_{min,n}
\end{align}
\begin{tabelle}
	\caption{Messwerte mit berechneten Differenzen für die 1. Messung ($500~Hz$)}
	\label{tab:auswertung->MittelwertResonanzabstand->Messung1}
	\sisetup{round-precision=1}
	\begin{tabular}{|c|c|c|c|c|c|c|c|c|c|c|c|}
		\hline \rowcolor{firstcsvrow}
		Messung & 1 & 2 & 3 & 4 & 5 & 6 & 7 & 8 & 9 & 10 & Mittelwert \\
		\csvreader[separator=semicolon, late after line = \\\hline]{./tables/A1a.csv}{}{
			\csvcoli & $\num{\csvcolii}$ & $\num{\csvcoliii}$ & $\num{\csvcoliv}$ & $\num{\csvcolv}$ & $\num{\csvcolvi}$ & $\num{\csvcolvii}$ & $\num{\csvcolviii}$ & $\num{\csvcolix}$ & $\num{\csvcolx}$ & $\num{\csvcolxi}$ & $\num{\csvcolxii}$
		}
	\end{tabular}
\end{tabelle}
\begin{tabelle}
	\caption{Messwerte mit berechneten Differenzen für die 2. Messung ($1000~Hz$)}
	\label{tab:auswertung->MittelwertResonanzabstand->Messung2}
	\sisetup{round-precision=1}
	\begin{tabular}{|c|c|c|c|c|c|c|c|c|c|c|c|}
		\hline \rowcolor{firstcsvrow}
		Messung & 1 & 2 & 3 & 4 & 5 & 6 & 7 & 8 & 9 & 10 & Mittelwert \\
		\csvreader[separator=semicolon, late after line = \\\hline]{./tables/A1b.csv}{}{
			\csvcoli & $\num{\csvcolii}$ & $\num{\csvcoliii}$ & $\num{\csvcoliv}$ & $\num{\csvcolv}$ & $\num{\csvcolvi}$ & $\num{\csvcolvii}$ & $\num{\csvcolviii}$ & $\num{\csvcolix}$ & $\num{\csvcolx}$ & $\num{\csvcolxi}$ & $\num{\csvcolxii}$
		}
	\end{tabular}
\end{tabelle}
\begin{tabelle}
	\caption{Messwerte mit berechneten Differenzen für die 3. Messung ($1500~Hz$)}
	\label{tab:auswertung->MittelwertResonanzabstand->Messung3}
	\sisetup{round-precision=1}
	\begin{tabular}{|c|c|c|c|c|c|c|c|c|c|c|c|}
		\hline \rowcolor{firstcsvrow}
		Messung & 1 & 2 & 3 & 4 & 5 & 6 & 7 & 8 & 9 & 10 & Mittelwert \\
		\csvreader[separator=semicolon, late after line = \\\hline]{./tables/A1c.csv}{}{
			\csvcoli & $\num{\csvcolii}$ & $\num{\csvcoliii}$ & $\num{\csvcoliv}$ & $\num{\csvcolv}$ & $\num{\csvcolvi}$ & $\num{\csvcolvii}$ & $\num{\csvcolviii}$ & $\num{\csvcolix}$ & $\num{\csvcolx}$ & $\num{\csvcolxi}$ & $\num{\csvcolxii}$
		}
	\end{tabular}
\end{tabelle}
\begin{tabelle}
	\caption{Messwerte mit berechneten Differenzen für die 4. Messung ($2000~Hz$)}
	\label{tab:auswertung->MittelwertResonanzabstand->Messung4}
	\sisetup{round-precision=1}
	\begin{tabular}{|c|c|c|c|c|c|c|c|c|c|c|c|}
		\hline \rowcolor{firstcsvrow}
		Messung & 1 & 2 & 3 & 4 & 5 & 6 & 7 & 8 & 9 & 10 & Mittelwert \\
		\csvreader[separator=semicolon, late after line = \\\hline]{./tables/A1d.csv}{}
			{\csvcoli & $\num{\csvcolii}$ & $\num{\csvcoliii}$ & $\num{\csvcoliv}$ & $\num{\csvcolv}$ & $\num{\csvcolvi}$ & $\num{\csvcolvii}$ & $\num{\csvcolviii}$ & $\num{\csvcolix}$ & $\num{\csvcolx}$ & $\num{\csvcolxi}$ & $\num{\csvcolxii}$}
	\end{tabular}
\end{tabelle}

\subsection{Berechnung der Wellenlänge $\lambda$}
\label{sub:auswertung->Wellenlaenge}
Abbildung \ref{ab:auswertung->Wellenlaenge->begruendung} zeigt die stehende Welle im Resonanzrohr mit den dazugehörigen Längenbeziehungen. Die Anzahl der Resonanzen ist in blau eingezeichnet, wobei bei der Resonanz $l_{min}$ mit eins zu zählen begonnen wurde.
\begin{abbildung}
	
\begin{tikzpicture}[scale=1.0, >=stealth']
    
    \draw[dashed] (0,1)node[left,font=\small] {$2s$} -- (11.8,1); //obere Begrenzung
    \draw[dashed] (0,-1)node[left,font=\small] {$-2s$} -- (11.8,-1); //untere Begrenzung
    \draw[->](0,-1.2)--(0,1.2); // y-Pfeil
    \foreach \x in {0,2,...,10}{
    \draw[black!65] (\x,-0.17) -- (\x,0.17);
    }
    \draw[white,text=red,font=\footnotesize] (0,0) -- ++(1,1) node[below]{1} -- ++(1,-1) -- ++(1,-1) node[above]{2} -- ++(1,1) -- ++(1,1) node[below]{3} -- ++(1,-1) -- ++(1,-1) node[above]{4} -- ++(1,1) -- ++(1,1) node[below]{5} -- ++(1,-1) -- ++(1,-1) node[above]{6} -- ++(1,1);
    
    \draw (1,0.2) -- (1,-0.2)node [below,font=\scriptsize,] {$l_{min}$};
    \draw (11,0.2) -- (11,-0.2)node [below,font=\scriptsize,] {$l_{max}$};
 	\draw[<->] (1,1.15) -- node[above]{$\lambda$}(5,1.15);
	\draw[<->] (1,-1.15) -- node[below]{$\Delta l_n=l_{max}-l_{min}$}(11,-1.15);
 	   
	\draw[very thick] (0,0) sin ++(1,1) cos ++(1,-1) sin ++(1,-1) cos ++(1,1) sin ++(1,1) cos ++(1,-1) sin ++(1,-1) cos ++(1,1) sin ++(1,1) cos ++(1,-1) sin ++(1,-1) cos ++(1,1);   
	\draw[thick, dotted, black!50] (0,0) sin ++(1,-1) cos ++(1,1) sin ++(1,1) cos ++(1,-1) sin ++(1,-1) cos ++(1,1) sin ++(1,1) cos ++(1,-1) sin ++(1,-1) cos ++(1,1) sin ++(1,1) cos ++(1,-1);
     
	\draw[->](0,0) -- (12.2,0) node[below,font=\footnotesize]{$l$};
     
	\draw[dashed] (-1.5,1.75) arc (45:-45:2.47487373415);  //2.47... = 3.5/sqrt(2)
	\draw[dashed] (14,-1.75) --  node[below](resonanzrohr){Resonanzrohr}(-1.5,-1.75) arc (225:135:2.47487373415) --(14,1.75);
	%\draw[<->] (resonanzrohr.south west) -- (resonanzrohr.south east);
	\node[rotate=-90,font=\scriptsize, fill=black!10, rounded corners] at(-1.5,0){Lautprecher};
	\draw[dotted] (12.3,-1.75)--(12.3,1.75);
	\node[rotate=0,font=\footnotesize, fill=black!0, rounded corners](wasser) at(13,0){Wasser}; 
	\draw[<->] (wasser.south west) -- (wasser.south east);
	% \draw (12,1.5) -- (12,-1.5);
\end{tikzpicture}
	\caption{Veranschaulichung der Stehenden Welle im Resonanzrohr mit Wellenparametern}
	\label{ab:auswertung->Wellenlaenge->begruendung}
\end{abbildung}
Aus der Abbildung lässt sich folgender Zusammenhang ableiten:
\begin{align}
\label{eq:auswertung->Wellenlaenge->berechnungvorschrift}
\lambda_n &= \frac{\Delta l_n}{n-1}\cdot 2
\end{align}
Mit den Mittelwerten der vier Messungen aus Abschnitt \ref{sub:auswertung->MittelwertResonanzabstand} und der Gleichung \ref{eq:auswertung->Wellenlaenge->berechnungvorschrift} wurde jeweils die Wellenlänge $\lambda$ berechnet und in Tabelle \ref{tab:auswertung->Wellenlaenge->Ergebnis} dargestellt.
\begin{tabelle}
	\caption{Berechnete Wellenlänge für die vier Messungen}
	\label{tab:auswertung->Wellenlaenge->Ergebnis}
	\sisetup{round-precision=2}
	\begin{tabular}{|c|c|c|c|}
		\hline \rowcolor{firstcsvrow}
		Messung & $\overline{\Delta l_n}$ in $cm$ & $n$ & $\lambda$ in $cm$ \\
		\csvreader[separator=semicolon, late after line = \\\hline]{./tables/Ergebnisse.csv}{}
			{\csvcoli~$(\csvcolii~Hz)$ & $\num{\csvcolvi}$ & $\csvcolv$ & $\num{\csvcolvii}$}
	\end{tabular}
\end{tabelle}

\subsection{Berechnung der Schallgeschwindigkeit $c$}
\label{sub:auswertung->Schallgeschwindigkeit}
Die Schallgeschwindigkeit $c$ wurde wie folgt berechnet und ist in Tabelle \ref{tab:auswertung->Schallgeschwindigkeit->Ergebnis} für jede Messung dargestellt.
\begin{align}
c_n&=f_n \cdot \lambda
\end{align}

\begin{tabelle}
	\caption{Ergebnisse der Berechnung der Schallgeschwindigkeit}
	\label{tab:auswertung->Schallgeschwindigkeit->Ergebnis}
	\sisetup{round-precision=1}
	\begin{tabular}{|c|c|c|c|}
		\hline \rowcolor{firstcsvrow}
		Messung & $f$ in $Hz$ & $\lambda$ in $cm$ & $c$ in $m/s$\\
		\csvreader[separator=semicolon, late after line = \\\hline]{./tables/Ergebnisse.csv}{}
			{\csvcoli & $\csvcolii$ & $\num{\csvcolvii}$ & $\num{\csvcolviii}$}
	\end{tabular}
\end{tabelle}

\subsection{Berücksichtigung des Temperatureinflusses}
\label{sub:auswertung->Temperatur}
Die Schallgeschwindigkeiten bei $20~\grad C$ der vier Messungen wurde mit Gleichung (11) berechnet und sind in Tabelle \ref{tab:auswertung->Temperatur->Ergebnis} 

\begin{tabelle}
	\caption{Werte der Schallgeschwindigkeit bei $20~\grad C$ der einzelnen Messungen}
	\label{tab:auswertung->Temperatur->Ergebnis}
	\sisetup{round-precision=2}
	\begin{tabular}{|c|c|c|c|c|}
		\hline \rowcolor{firstcsvrow}
		Messung &  $T_v$ in $\grad C$ & $T_n$ in $\grad C$ & $c$ in $m/s$ & $c_{exp}(20~\grad C)$ in $m/s$\\
		\csvreader[separator=semicolon, late after line = \\\hline]{./tables/Ergebnisse.csv}{}
			{\csvcoli~$(\csvcolii~Hz)$ & $\num[round-precision=1]{\csvcoliii}$ & $\num[round-precision=1]{\csvcoliv}$ & $\num{\csvcolviii}$ & $\num{\csvcolix}$}
	\end{tabular}
\end{tabelle}

\subsection{Zusammenfassung der Ergebnisse und Vergleich mit Literaturwert}
\label{sub:auswertung->Zusammenfassung}
Tabelle \ref{tab:auswertung->zusammenfassung->vergleich} zeigt den Vergleich des experimentellen Wertes $c_{exp}$ mit dem Literaturwert $c_{lit}=\num{343,14}$ in Luft bei $20 ~\grad C$.
\begin{tabelle}
	\caption{Vergleich der Messwerte mit dem Literaturwert bei 20 $\grad C$}
	\label{tab:auswertung->zusammenfassung->vergleich}
	\sisetup{round-precision=3}
	\begin{tabular}{|c|c|c|c|c|}
		\hline \rowcolor{firstcsvrow}
		Messung & $c_{exp}(20~\grad C)$ & $c_{lit}(20~\grad C)$ & abs. Abweichung & rel. Abweichung\\
		\csvreader[separator=semicolon, late after line = \\\hline]{./tables/Ergebnisse.csv}{}
			{\csvcoli~$(\csvcolii~Hz)$ & $\num{\csvcolix}~m/s$ & $\num[round-precision=2]{\csvcolx}~m/s$ & $\num{\csvcolxi}~m/s$ & $\num{\csvcolxii} ~\%$}
		Mittelwert& $\num{333,7448981}~m/s$	& $\num[round-precision=2]{343,14}~m/s$	& $\num{-9,395101917}~m/s$	& $\num{-2,737979226}~\%$
		\\\hline
	\end{tabular}
\end{tabelle}

\subsection{Diskussion der Ergebnisse}
\label{sub:auswertung->Diskussion}
Betrachtet man Tabelle \ref{tab:auswertung->zusammenfassung->vergleich} fällt auf, dass mit steigender Frequenz die Abweichung vom Literaturwert größer wird. Dies liegt wahrscheinlich daran, dass die Bandbreite der einzelnen Resonanzfrequenz geringer wird, weil mehr Resonanzen auf der Länge des Resonanzrohres zu messen sind, und somit die Wellenbäuche spitzer und schmaler werden. Daher war die Stelle der Auslenkungsmaxima präziser abzulesen. \\
%Die sehr geringe Abweichung bei der vierten Messung sieht schön aus, ist allerdings aber mit Vorsicht zu betrachten, da  \\
Ein weiterer Grund für Fehler trat beim Ablesen auf. Hier war es schwierig Werte für $l_{max}$ genau zu bestimmen, da man nicht direkt im rechten Winkel auf den Maßstab schauen konnte, sondern von schräg oben ablesen musste, weil das mit Wasser gefüllte Rohr im Weg war. Dabei schätzen wir eine Messungenauigkeit von $1-2~mm$. 
Für die Messung wäre auch eine ausführliche Fehlerrechnung möglich, um so die Ergebnisse besser zu beurteilen.
%========Marius=================
	\newpage
\section{Geräteliste}
\label{sec:geraeteliste}
\begin{tabelle}
	\caption{Geräteliste}
	\label{tab:geraeteliste}
	\begin{tabular}{|c|K{3.5\tablewidth}|}
		\hline \rowcolor{firstcsvrow}
		Nr. & Gerät \\
		\hline
		\csvreader[separator= semicolon, late after line=\\\hline]
			{tables/geraeteliste.csv}{}{\thecsvrow & \csvcoli}
	%	\leereZellen[10]{\csvcoli & }
		
	\end{tabular}
\end{tabelle}
%Literatur- und Abkürzungsverzeichnis 	
\def\usecounter#1{\@nmbrlisttrue\def\@listctr{#1}\setcounter{#1}{0}}
\bibliography{bib/s4}
%\addcontentsline{toc}{section}{Literatur}
\section*{Anhang}
\addcontentsline{toc}{section}{Anhang}
%\setcounter{section}{0}
%Messprotokoll
\label{letzteSeite}
\section{Messprotokoll}
\label{sec:messprotokoll}
\pagestyle{messprotokoll}


%======================================%
%================Inhalt================%

 
\newpage
\kariert{47}
\section{Vorbereitungsstichpunkte}
\label{subsec:vorbereitung->stichpunkte}
%Abkürzungen
%\section*{Abkürzungen}
\begin{acronym}[längste Abkürzung]
	\acro{Abkürzung}{Ausgeschrieben}
	\acroplural{Abkürzung}[AbkürzungPlural]{abweichender ausgeschr. Plural}
\end{acronym}
\end{document}