%==============================================%
%==============Diagrammdefinition==============%
%Autor: 		Nico Herkner
%Erstelldatum:	
%Version:
%		1.2		2017-11-01	NH
%			NH	added:	\leereZellen[2]
%			NH	added:	\kariert[2]
%			NH	added:	\linie[1]
%			NH	changed:\linie[2] 
%			NH	changed:\leereZellen
%		1.3		2017-11-04	NH
%			NH	changed:\data
%			NH	changed:\data
%			NH	added:	\pgfplotsset
%			NH	added:	\MyAxis style
%			NH	added:	\MyPlotCsv style
%		1.4		2017-11-15	NH
%			NH	added:	\ome
%			NH	added:	\grad
%			NH	added:	Längen


%Settings for Tikz and pgfplots
\pgfplotsset{
	MyAxis/.style={
		/pgf/number format/.cd, use comma,
		1000 sep={.},
		anchor=origin,	% Align the origins
		%x=1cm, y=1cm,	% Set the same unit vectors
		grid=major, % erstellt ein Raster
		width=\linewidth,
		cycle list name=MyCyclelist,
	},
	MyPlotCsv/.style={
		/pgf/number format/read comma as period, %Komma statt Punkt als Dezimaltrennzeichen
		domain=0:360,
		samples=100
	},
}


%Counter
\newcounter{owncsvrow}
\newcounter{counterA}
\newcounter{counterB}
\newcounter{counterC}
\newcounter{counterD}
\newcounter{counterE}
\newcounter{counterF}

%neue Definitionen
\newcommand{\ome}{\omega}
\newcommand{\grad}{^\circ}
\newcommand{\leer}[0]{ } % Leertaste
\newcommand{\fett}[1]{\textbf{#1}}%Fett
\newcommand{\kursiv}[1]{\\textit{#1}}%Kursiv
\newcommand{\unterstrichen}[1]{\textbf{#1}}%Unterstruchen
%Diagramm
\newcommand{\data}[3][]{
	\addplot
		table[
			col sep =semicolon, %Trennzeichen zwischen Spalten
			/pgf/number format/read comma as period, %Komma statt Punkt als Dezimaltrennzeichen
			y=#3, %Spaltenname für y-Werte
			domain=0:15,
    		samples=100,
			#1
		]{#2};
}
\newcommand{\diagramm}[3]{
	\begin{tikzpicture}[show background rectangle]
		\begin{axis}[
			%smooth,
			%axis x line=bottom, axis y line=left, %Pfeile bei den Achsen
			width=\linewidth, %passt die breite des Diagramms auf die Seitenbreite an
			grid=major, % erstellt ein Raster
			grid style={dashed,gray!30}, %formatiert das Raster
			cycle list name=MyCyclelist,
			xlabel=$#1$\newline test, %Erstellt eine Achsenbeschriftung
			ylabel=$#2$,
			%legend style={at={(0.5,-0.2)},anchor=north}, %Positioniert die Legende unter der Tabelle
			]
		#3 %Daten
		\end{axis}
	\end{tikzpicture}
}
%Leere Zeilen in einer Tabelle \leereZeilen{ZeilenAnzahl}{Aufbau einer Zeile mit &}
\newcommand{\leereZellen}[2]{
	\hline
	\csvreader[separator = semicolon, late after line =\\\hline, filter test=\ifnumless{\csvcoli}{#1+1}]{./tables/hundert.csv}{}{#2}
}
%Karriert \kariert[steplength]{zeilen}
\newcommand\kariert[2][0.5cm]{% 
   \begin{tikzpicture}[gray,step=#1]
     \pgfmathtruncatemacro\anzahl{(\linewidth-\pgflinewidth)/#1} % maximale Anzahl Kästchen pro Zeile
     \draw (0,0) rectangle (\anzahl*#1,#2*#1) (0,0) grid (\anzahl*#1,#2*#1);
   \end{tikzpicture} 
}
\newcommand{\linie}[2][\textwidth]{
	\foreach \x in {1,...,#2} {\\[9pt] \noindent\rule{#1}{1pt}}
}

%Ergebnis doppelt unterstrichen
\newcommand{\result}[1]{
	\underline{\underline{#1}}
}
%Einheit
\newcommand{\unit}[1]{
	~\mathrm{#1}
}
%Unterstrichen
\renewcommand{\u}[1]{
	\underline{#1}
}
%Bruch
\renewcommand{\/}[2]{
	~\frac{#1}{#2}
}
%Bruch
\newcommand{\fraq}[2]{
	~\frac{#1}{#2}
}
%Gleichung
\newcommand{\eq}[1]{
	(\ref{#1})
}
%Varphi
\newcommand{\fii}[0]{
	\varphi \alpha
}
%Vektor
\newcommand{\vek}[1]{
	\overset{\rightharpoonup}{#1}
}
%Zahlenmengen
\newcommand{\R}{\mathbb{R}}
\newcommand{\C}{\mathbb{C}}
\newcommand{\N}{\mathbb{N}}
\newcommand{\G}{\mathbb{G}}
\newcommand{\Q}{\mathbb{Q}}
%Betragsstriche
\providecommand{\abs}[1]{
	\lvert #1 \rvert
}


%Neue Umgebungen
%Tabelle
\newenvironment{tabele}{\begin{longtable}}{\end{longtable}}
\newenvironment{tabelle}{\begin{table}[H] \centering}{\end{table}}
\newenvironment{abbildung}{\begin{figure}[H] \centering}{\end{figure}}


%Neue Farben
\definecolor{darkspringgreen}{rgb}{0.09, 0.45, 0.27}
\definecolor{darkpastelred}{rgb}{0.76, 0.23, 0.13}
\definecolor{dimgray}{rgb}{0.41, 0.41, 0.41}
\definecolor{firstcsvrow}{rgb}{0.68, 0.68, 1}
\definecolor{secondcsvrow}{rgb}{0.75, 0.75, 0.75}

%Neuer Spaltentyp 
\newcolumntype{K}[1]{>{\centering\arraybackslash}p{#1}} %Mittig mit bestimmter Breie

%Farbreihenfolge bei Plots
\pgfplotscreateplotcyclelist{MyCyclelist}{%
  {red, mark = *, thick},
  {blue, mark = square*, thick},
  {orange, mark = diamond*, thick},
  {green, mark = pentagon*, thick},
  {yellow, mark = *, thick},
  {black, mark = triangle*, thick},
}
%legend image post style={line width=1pt}

%Längen
\newlength{\tablewidth}
\setlength{\tablewidth}{1.5cm}
\setlength{\parindent}{0em} %einrücken verhindern